\documentclass{article}

\title{Music Box Games Programming Style Guide}
\date{2019-09-25}
\author{Ryan Hanson}

\usepackage{graphicx}
\usepackage{setspace}
\usepackage{listings}
\usepackage{color}
\usepackage[dvipsnames]{xcolor}
\usepackage{hyperref}
\usepackage{fancyhdr}

% set up footer
\pagestyle{fancy}
\cfoot{Copyright (c) 2019 Music Box Games}

% setup listings package for source code highlighting
\lstset{
language=c++,
basicstyle=\small\sffamily,
numbers=left,
numberstyle=\tiny,
frame=tb,
tabsize=4,
columns=fixed,
showstringspaces=false,
showtabs=false,
keepspaces,
commentstyle=\color{ForestGreen},
keywordstyle=\color{blue}
}

% define variables
\newcommand{\cxxlangstd}{C++ 17}


\begin{document}
\pagenumbering{gobble}
\maketitle
\newpage
\tableofcontents
\newpage
\pagenumbering{arabic}

\section*{Overview}
This document outlines the required programming style for development at Music Box Games. This is an internal document and is not to be shared.

\section{Source Control}
\subsection{Git}
\paragraph{}
All code should be regularly checked into the development git repository on a branch given an appropriate name (If you're working on graphics, the branch should be named graphics). Working in master is only allowed with permission and final merges of code should be done by the project's Technical Director after sufficient testing.

\section{Commenting}
\subsection{C++}
\paragraph*{}
Documentation for C++ source code should be done with Doxygen style comments. Every source and header file must contain a header comment, and each function should have their own header comment.

\paragraph*{}
A good file header comment should explain the purpose of code in the file in as much detail as necessary:
\lstinputlisting{src/hdrcmt.cpp}
\newpage
\paragraph*{}
Function header comments should give a more detailed explanation than the file header, as it is specific to only one function. Details on the inputs, outputs, and functionality of function.
\lstinputlisting{src/fnccmt.cpp}
\paragraph*{}
Additional info such as run-time complexity can also be added at your discretion.
\newline
For more information of Doxygen style commenting, \href{http://www.doxygen.nl/manual/index.html}{see here}

\section{Language}
\paragraph*{}
You should always program using modern, but reliable conventions.
\subsection{C++}
\paragraph*{}
The current approved standard for C++ is \cxxlangstd{}, this will be updated when a newer standard is supported on all operating systems. Please use as much modern C++ as you can.

\section{Build Systems}
\paragraph*{}
Please use the build systems provided in the development repositories. For more specific details on build systems, see the Project Primer document present in the repository.

\end{document}
